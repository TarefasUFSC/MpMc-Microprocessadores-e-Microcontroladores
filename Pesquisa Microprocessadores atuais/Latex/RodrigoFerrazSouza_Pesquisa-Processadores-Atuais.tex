% ------------------------------------------------------------------------
% ------------------------------------------------------------------------
% Modelo UFSC para Trabalhos Academicos (tese de doutorado, dissertação de
% mestrado) utilizando a classe abntex2
%
% Autor: Alisson Lopes Furlani
% 	Modificações:
%	- 27/08/2019: Alisson L. Furlani, add pacote 'glossaries' para listas
%   - 06/11/2019: Luiz-Rafael Santos, modifica para Trabalho de Conclusão de Curso
% ------------------------------------------------------------------------
% ------------------------------------------------------------------------

\documentclass[
	% -- opções da classe memoir --
	12pt,				% tamanho da fonte
	%openright,			% capítulos começam em pág ímpar (insere página vazia caso preciso)
	oneside,			% para impressão no anverso. Oposto a twoside
	a4paper,			% tamanho do papel. 
	% -- opções da classe abntex2 --
	chapter=TITLE,		% títulos de capítulos convertidos em letras maiúsculas
	section=TITLE,		% títulos de seções convertidos em letras maiúsculas
	%subsection=TITLE,	% títulos de subseções convertidos em letras maiúsculas
	%subsubsection=TITLE,% títulos de subsubseções convertidos em letras maiúsculas
	% -- opções do pacote babel --
	english,			% idioma adicional para hifenização
	%french,				% idioma adicional para hifenização
	%spanish,			% idioma adicional para hifenização
	brazil				% o último idioma é o principal do documento
	]{abntex2}

\usepackage{setup/ufscthesisA4-alf}

% ---
% Filtering and Mapping Bibliographies
% ---
% Pacotes de citações
% ---
\usepackage{csquotes}
\usepackage{multirow}
\usepackage[table,xcdraw]{xcolor}
\usepackage{float}
\usepackage[backend = biber, style = abnt]{biblatex}
% FIXME Se desejar estilo numérico de citações,  comente a linha acima e descomente a linha a seguir.
% \usepackage[backend = biber, style = numeric-comp]{biblatex}

\setlength\bibitemsep{\baselineskip}
\DeclareFieldFormat{url}{Disponível~em:\addspace\url{#1}}
\NewBibliographyString{sineloco}
\NewBibliographyString{sinenomine}
\DefineBibliographyStrings{brazil}{%
	sineloco     = {\mkbibemph{S\adddot l\adddot}},
	sinenomine   = {\mkbibemph{s\adddot n\adddot}},
	andothers    = {\mkbibemph{et\addabbrvspace al\adddot}},
	in			 = {\mkbibemph{In:}}
}

\addbibresource{aftertext/references.bib} % Seus arquivos de referências

% ---
\DeclareSourcemap{
	\maps[datatype=bibtex]{
		% remove fields that are always useless
		\map{
			\step[fieldset=abstract, null]
			\step[fieldset=pagetotal, null]
		}
		% remove URLs for types that are primarily printed
%		\map{
%			\pernottype{software}
%			\pernottype{online}
%			\pernottype{report}
%			\pernottype{techreport}
%			\pernottype{standard}
%			\pernottype{manual}
%			\pernottype{misc}
%			\step[fieldset=url, null]
%			\step[fieldset=urldate, null]
%		}
		\map{
			\pertype{inproceedings}
			% remove mostly redundant conference information
			\step[fieldset=venue, null]
			\step[fieldset=eventdate, null]
			\step[fieldset=eventtitle, null]
			% do not show ISBN for proceedings
			\step[fieldset=isbn, null]
			% Citavi bug
			\step[fieldset=volume, null]
		}
	}
}
% ---

% ---
% Informações de dados para CAPA e FOLHA DE ROSTO
% ---
% FIXME Substituir 'Nome completo do autor' pelo seu nome.
\autor{Rodrigo Ferraz Souza (19103563)}
% FIXME Substituir 'Título do trabalho' pelo título da trabalho.
\titulo{Pesquisa Sobre Processadores Atuais}
% FIXME Substituir 'Subtítulo (se houver)' pelo subtítulo da trabalho.  
% Caso não tenha substítulo, comente a linha a seguir.
% FIXME Substituir 'XXXXXX' pelo nome do seu
% orientador.
\orientador{Prof. XXXXXX, Dr.}
% FIXME Se for orientado por uma mulher, comente a linha acima e descomente a linha a seguir.
% \orientador[Orientadora]{Nome da orientadora, Dra.}
% FIXME Substituir 'XXXXXX' pelo nome do seu
% coorientador. Caso não tenha coorientador, comente a linha a seguir.
\coorientador{Prof. XXXXXX, Dr.}
% FIXME Se for coorientado por uma mulher, comente a linha acima e descomente a linha a seguir.
% \coorientador[Coorientadora]{XXXXXX, Dra.}
% FIXME Substituir 'XXXXXX' pelo nome do Coordenador do 
% programa/curso.
\coordenador{Prof. XXXXXX, Dr.}
% FIXME Se for coordenadora mulher, comente a linha acima e descomente a linha a seguir.
% \coordenador[Coordenadora]{Nome da Coordenadora, Dra.}
% FIXME Substituir '[ano da entrega]' pelo ano (ano) em que seu trabalho foi defendido.
\ano{2021}
% FIXME Substituir '[dia] de [mês] de [ano]' pela data em que ocorreu sua defesa.
\data{[25] de [Outubro] de [2021]}
% FIXME Substituir '[Cidade da defesa]' pela cidade em que ocorreu sua defesa.
\local{Araranguá}
\instituicaosigla{UFSC}
\instituicao{Universidade Federal de Santa Catarina}
% FIXME Substituir 'Dissertação/Tese' pelo tipo de trabalho (Tese, Dissertação). 
\tipotrabalho{Trabalho de Conclusão de Curso}
% FIXME Substituir '[licenciado/bacharel] em [nome do título obtido]' pela grau adequado.
\formacao{[licenciado/bacharel] em [nome do título obtido]}
% FIXME Substituir '[licenciado/bacharel]' pelo nivel adequado.
\nivel{[licenciado/bacharel]}
% FIXME Substituir 'Curso de Graduação em [XXXXXXXX]' pela curso adequado.
\programa{Curso de Graduação em Engenharia da Computação}
% FIXME Substituir 'Campus XXXXXX ou Centro de XXXXXX' pelo campus ou centro adequado.
\centro{Campus Araranguá}
\preambulo
{%
\imprimirtipotrabalho~do~\imprimirprograma~do~\imprimircentro~da~\imprimirinstituicao~para~a~obtenção~do~título~de~\imprimirformacao.
}
% ---

% ---
% Configurações de aparência do PDF final
% ---
% alterando o aspecto da cor azul
\definecolor{blue}{RGB}{41,5,195}
% informações do PDF
\makeatletter
\hypersetup{
     	%pagebackref=true,
		pdftitle={\@title}, 
		pdfauthor={\@author},
    	pdfsubject={\imprimirpreambulo},
	    pdfcreator={LaTeX with abnTeX2},
		pdfkeywords={ufsc, latex, abntex2}, 
		colorlinks=true,       		% false: boxed links; true: colored links
    	linkcolor=black,%blue,          	% color of internal links
    	citecolor=black,%blue,        		% color of links to bibliography
    	filecolor=black,%magenta,      		% color of file links
		urlcolor=black,%blue,
		bookmarksdepth=4
}
\makeatother
% ---

% ---
% compila a lista de abreviaturas e siglas e a lista de símbolos
% ---

% Declaração das siglas
\siglalista{ABNT}{Associação Brasileira de Normas Técnicas}

% Declaração dos simbolos
\simbololista{C}{\ensuremath{C}}{Circunferência de um círculo}
\simbololista{pi}{\ensuremath{\pi}}{Número pi} 
\simbololista{r}{\ensuremath{r}}{Raio de um círculo}
\simbololista{A}{\ensuremath{A}}{Área de um círculo}

% compila a lista de abreviaturas e siglas e a lista de símbolos
\makenoidxglossaries 

% ---

% ---
% compila o indice
% ---
\makeindex
% ---

% ----
% Início do documento
% ----
\begin{document}

% Seleciona o idioma do documento (conforme pacotes do babel)
%\selectlanguage{english}
\selectlanguage{brazil}

% Retira espaço extra obsoleto entre as frases.
\frenchspacing

% Espaçamento 1.5 entre linhas
\OnehalfSpacing

% Corrige justificação
%\sloppy

% ----------------------------------------------------------
% ELEMENTOS PRÉ-TEXTUAIS
% ----------------------------------------------------------
% \pretextual %a macro \pretextual é acionado automaticamente no início de \begin{document}
% ---
% Capa, folha de rosto, ficha bibliografica, errata, folha de apróvação
% Dedicatória, agradecimentos, epígrafe, resumos, listas
% ---
% ---
% Capa
% ---
\imprimircapa
% ---



{%hidelinks
	\hypersetup{hidelinks}

	% ---
	% inserir o sumario
	% ---
	\pdfbookmark[0]{\contentsname}{toc}
	\tableofcontents*
	\cleardoublepage

}%hidelinks
% ---
% ---

% ----------------------------------------------------------
% ELEMENTOS TEXTUAIS
% ----------------------------------------------------------
\textual

% ---
% 1 - Introdução
% ---
% ----------------------------------------------------------
\chapter{Introdução}
% ----------------------------------------------------------


Este trabalho se dedica a apresentar brevemente algumas carateristas dos modelos mais atuais de processadores no semestre 21.2. Aqui também é discutido brevemente algumas de suas tecnologias e uma comparação entre as marcas pesquisadas.
% ---

% ---
% 2 - Capítulo 2
% ---
% ----------------------------------------------------------
\chapter{Pesquisa}\label{cap:pesquisa}
% ----------------------------------------------------------
Aqui vão ser apresentadas os resultados das pesquisas referentes ao mais recentes modelos da Intel e da AMD


% ----------------------------------------------------------
\section{AMD Ryzen 9 5900X}
% ----------------------------------------------------------

\begin{table}[!h]
	\centering
	\begin{tabular}{|c|l|c|}
		\hline
		\rowcolor[HTML]{BDBDBD}
		\multicolumn{2}{|c|}{\cellcolor[HTML]{BDBDBD}Especificação}               & Descrição                                                                             \\ \hline
		\rowcolor[HTML]{FFFFFF}
		\cellcolor[HTML]{FFFFFF}                                                  & \multicolumn{1}{c|}{\cellcolor[HTML]{FFFFFF}Físicos}                & 12              \\ \cline{2-3}
		\rowcolor[HTML]{F3F3F3}
		\multirow{-2}{*}{\cellcolor[HTML]{FFFFFF}Número de Núcleos}               & \multicolumn{1}{c|}{\cellcolor[HTML]{F3F3F3}Threads}                & 24              \\ \hline
		\rowcolor[HTML]{FFFFFF}
		\multicolumn{2}{|c|}{\cellcolor[HTML]{FFFFFF}Tamanho da Palavra de Dados} & 64-bits                                                                               \\ \hline
		\rowcolor[HTML]{F3F3F3}
		\multicolumn{2}{|c|}{\cellcolor[HTML]{F3F3F3}GPU Integrada}               & Não Possui                                                                            \\ \hline
		\rowcolor[HTML]{FFFFFF}
		\cellcolor[HTML]{FFFFFF}                                                  & \multicolumn{1}{c|}{\cellcolor[HTML]{FFFFFF}Max Boost}              & até 4,8GHz      \\ \cline{2-3}
		\rowcolor[HTML]{F3F3F3}
		\multirow{-2}{*}{\cellcolor[HTML]{FFFFFF}Clock}                           & \multicolumn{1}{c|}{\cellcolor[HTML]{F3F3F3}Base}                   & 3,7GHz          \\ \hline
		\rowcolor[HTML]{FFFFFF}
		\multicolumn{2}{|c|}{\cellcolor[HTML]{FFFFFF}Número de Transistores}      & 4,15 Bilhões                                                                          \\ \hline
		\rowcolor[HTML]{F3F3F3}
		\cellcolor[HTML]{F3F3F3}                                                  & \multicolumn{1}{c|}{\cellcolor[HTML]{F3F3F3}Nome}                   & Zen3            \\ \cline{2-3}
		\rowcolor[HTML]{FFFFFF}
		\cellcolor[HTML]{F3F3F3}                                                  & \multicolumn{1}{c|}{\cellcolor[HTML]{FFFFFF}Conjunto de Instruções} & CISC            \\ \cline{2-3}
		\rowcolor[HTML]{F3F3F3}
		\multirow{-3}{*}{\cellcolor[HTML]{F3F3F3}Arquitetura}                     & \multicolumn{1}{c|}{\cellcolor[HTML]{F3F3F3}Litografia}             & TSMC 7nm FinFET \\ \hline
		\rowcolor[HTML]{FFFFFF}
		\cellcolor[HTML]{FFFFFF}                                                  & \multicolumn{1}{c|}{\cellcolor[HTML]{FFFFFF}L2}                     & 6MB             \\ \cline{2-3}
		\rowcolor[HTML]{F3F3F3}
		\multirow{-2}{*}{\cellcolor[HTML]{FFFFFF}Cachê}                           & \multicolumn{1}{c|}{\cellcolor[HTML]{F3F3F3}L3}                     & 64MB            \\ \hline
		\rowcolor[HTML]{FFFFFF}
		\multicolumn{2}{|c|}{\cellcolor[HTML]{FFFFFF}TDP}                         & 105W                                                                                  \\ \hline
		\rowcolor[HTML]{F3F3F3}
		\multicolumn{2}{|c|}{\cellcolor[HTML]{F3F3F3}Família de Processadores}    & AMD Ryzen Processors                                                                  \\ \hline
		\rowcolor[HTML]{FFFFFF}
		\multicolumn{2}{|c|}{\cellcolor[HTML]{FFFFFF}Temperatura Maxima}          & 90°C                                                                                  \\ \hline
		\rowcolor[HTML]{F3F3F3}
		\multicolumn{2}{|c|}{\cellcolor[HTML]{F3F3F3}Tamanho do CI (die)}         & 80,7mm²                                                                               \\ \hline
		\rowcolor[HTML]{FFFFFF}
		\multicolumn{2}{|c|}{\cellcolor[HTML]{FFFFFF}Data de Lançamento}          & 05/11/2020                                                                            \\ \hline
	\end{tabular}
	\caption{Tabela de Especificações AMD Ryzen 9 5900X}
	\cite{AdrenaAMD}\cite{siteAMD}\cite{TechAMD}\cite{tomsAMD}\cite{RISCCISC}
	\label{table:amd}
\end{table}




% ----------------------------------------------------------
\section{Intel i9-11900K}
% ----------------------------------------------------------

\begin{table}[!h]
	\centering
	\begin{tabular}{|c|c|c|}
		\hline
		\rowcolor[HTML]{BDBDBD}
		\multicolumn{2}{|c|}{\cellcolor[HTML]{BDBDBD}Especificação}               & Descrição                                                               \\ \hline
		\rowcolor[HTML]{FFFFFF}
		\cellcolor[HTML]{FFFFFF}                                                  & Físicos                     & 8                                         \\ \cline{2-3}
		\rowcolor[HTML]{F3F3F3}
		\multirow{-2}{*}{\cellcolor[HTML]{FFFFFF}Número de Núcleos}               & Threads                     & 16                                        \\ \hline
		\rowcolor[HTML]{FFFFFF}
		\multicolumn{2}{|c|}{\cellcolor[HTML]{FFFFFF}Tamanho da Palavra de Dados} & 64-bits                                                                 \\ \hline
		\rowcolor[HTML]{F3F3F3}
		\multicolumn{2}{|c|}{\cellcolor[HTML]{F3F3F3}GPU Integrada}               & Intel UHD Graphics 750                                                  \\ \hline
		\rowcolor[HTML]{FFFFFF}
		\cellcolor[HTML]{FFFFFF}                                                  & Max Boost                   & 5,3GHz                                    \\ \cline{2-3}
		\rowcolor[HTML]{F3F3F3}
		\multirow{-2}{*}{\cellcolor[HTML]{FFFFFF}Clock}                           & Base                        & 3,5GHz                                    \\ \hline
		\rowcolor[HTML]{FFFFFF}
		\multicolumn{2}{|c|}{\cellcolor[HTML]{FFFFFF}Número de Transistores}      & Não informado pela empresa                                              \\ \hline
		\rowcolor[HTML]{F3F3F3}
		\cellcolor[HTML]{F3F3F3}                                                  & Nome                        & Rocket Lake                               \\ \cline{2-3}
		\rowcolor[HTML]{FFFFFF}
		\cellcolor[HTML]{F3F3F3}                                                  & Conjunto de Instruções      & CISC                                      \\ \cline{2-3}
		\rowcolor[HTML]{F3F3F3}
		\multirow{-3}{*}{\cellcolor[HTML]{F3F3F3}Arquitetura}                     & Litografia                  & 14nm                                      \\ \hline
		\rowcolor[HTML]{FFFFFF}
		\cellcolor[HTML]{FFFFFF}                                                  & Tamanho                     & 16MB                                      \\ \cline{2-3}
		\rowcolor[HTML]{F3F3F3}
		\multirow{-2}{*}{\cellcolor[HTML]{FFFFFF}Cachê}                           & Tecnologia de Gerenciamento & Intel Smart Cache (Ver \ref{cap:smcache}) \\ \hline
		\rowcolor[HTML]{FFFFFF}
		\multicolumn{2}{|c|}{\cellcolor[HTML]{FFFFFF}TDP}                         & 95W                                                                     \\ \hline
		\rowcolor[HTML]{F3F3F3}
		\multicolumn{2}{|c|}{\cellcolor[HTML]{F3F3F3}Família de Processadores}    & 11ª Gração i9                                                           \\ \hline
		\rowcolor[HTML]{FFFFFF}
		\multicolumn{2}{|c|}{\cellcolor[HTML]{FFFFFF}Temperatura Maxima}          & 100°C                                                                   \\ \hline
		\rowcolor[HTML]{F3F3F3}
		\multicolumn{2}{|c|}{\cellcolor[HTML]{F3F3F3}Tamanho do CI (die)}         & 37,5mm²                                                                 \\ \hline
		\rowcolor[HTML]{FFFFFF}
		\multicolumn{2}{|c|}{\cellcolor[HTML]{FFFFFF}Data de Lançamento}          & 1º Quadrimestre de 2021                                                 \\ \hline
	\end{tabular}
	\caption{Tabela de Especificações intel core i9-11900K \cite{intel}}
	\label{table:intel}
\end{table}

\subsection{Smart Cache}\label{cap:smcache}

Smart Cache é um método de cache de nível 2 ou nível 3 para núcleos de execução múltipla, desenvolvido pela Intel.O Smart Cache compartilha a memória cache real entre os núcleos de um processador multi-core. Em comparação com um cache per-core dedicado, a taxa geral de falta de cache diminui quando nem todos os núcleos precisam de partes iguais do espaço do cache. Consequentemente, um único núcleo pode usar o cache de nível 2 ou nível 3, se os outros núcleos estiverem inativos\cite{wiki}
% ---

% ---
% 2 - Capítulo 3
% ---
\chapter{Conclusão}\label{cap:concs}

Como é observado a partir da comparação entre as tabelas apresentadas, é nítida uma distinção entre os objetivos das fabricantes ao desenvolver os processadores aqui abordados. A Intel, por ter um publico extremamente amplo, focou em um processador top de linha mais genérico tentando cobrir a maioria das pontas soltas, como eficiência energética e poder computacional, contudo, ao tentar ser generalista não consegue competir em certas áreas.


Já a AMD consegue cobrir o nicho de mercado deixado pela Intel, como dito acima, pois no AMD Ryzen 9 5900K é claro, inclusive na propaganda, que o processador tem o foco no público gamer, que não dá tanta relevância a eficiência energética, mas sim para a velocidade que o processador consegue resolver ser problemas nos games. Por isso as especificações da AMD no modelo abordado tendem a ser mais fortes que a Intel neste quesito.
% ---

% ----------------------------------------------------------
% ELEMENTOS PÓS-TEXTUAIS
% ----------------------------------------------------------
\postextual
% ----------------------------------------------------------

% ----------------------------------------------------------
% Referências bibliográficas
% ----------------------------------------------------------
\begingroup
\SingleSpacing\printbibliography[title=REFERÊNCIAS]
\endgroup


\end{document}