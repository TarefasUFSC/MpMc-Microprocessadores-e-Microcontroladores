% ----------------------------------------------------------
\chapter{Pesquisa}\label{cap:pesquisa}
% ----------------------------------------------------------
Aqui vão ser apresentadas os resultados das pesquisas referentes ao mais recentes modelos da Intel e da AMD


% ----------------------------------------------------------
\section{AMD Ryzen 9 5900X}
% ----------------------------------------------------------

\begin{table}[!h]
	\centering
	\begin{tabular}{|c|l|c|}
		\hline
		\rowcolor[HTML]{BDBDBD}
		\multicolumn{2}{|c|}{\cellcolor[HTML]{BDBDBD}Especificação}               & Descrição                                                                             \\ \hline
		\rowcolor[HTML]{FFFFFF}
		\cellcolor[HTML]{FFFFFF}                                                  & \multicolumn{1}{c|}{\cellcolor[HTML]{FFFFFF}Físicos}                & 12              \\ \cline{2-3}
		\rowcolor[HTML]{F3F3F3}
		\multirow{-2}{*}{\cellcolor[HTML]{FFFFFF}Número de Núcleos}               & \multicolumn{1}{c|}{\cellcolor[HTML]{F3F3F3}Threads}                & 24              \\ \hline
		\rowcolor[HTML]{FFFFFF}
		\multicolumn{2}{|c|}{\cellcolor[HTML]{FFFFFF}Tamanho da Palavra de Dados} & 64-bits                                                                               \\ \hline
		\rowcolor[HTML]{F3F3F3}
		\multicolumn{2}{|c|}{\cellcolor[HTML]{F3F3F3}GPU Integrada}               & Não Possui                                                                            \\ \hline
		\rowcolor[HTML]{FFFFFF}
		\cellcolor[HTML]{FFFFFF}                                                  & \multicolumn{1}{c|}{\cellcolor[HTML]{FFFFFF}Max Boost}              & até 4,8GHz      \\ \cline{2-3}
		\rowcolor[HTML]{F3F3F3}
		\multirow{-2}{*}{\cellcolor[HTML]{FFFFFF}Clock}                           & \multicolumn{1}{c|}{\cellcolor[HTML]{F3F3F3}Base}                   & 3,7GHz          \\ \hline
		\rowcolor[HTML]{FFFFFF}
		\multicolumn{2}{|c|}{\cellcolor[HTML]{FFFFFF}Número de Transistores}      & 4,15 Bilhões                                                                          \\ \hline
		\rowcolor[HTML]{F3F3F3}
		\cellcolor[HTML]{F3F3F3}                                                  & \multicolumn{1}{c|}{\cellcolor[HTML]{F3F3F3}Nome}                   & Zen3            \\ \cline{2-3}
		\rowcolor[HTML]{FFFFFF}
		\cellcolor[HTML]{F3F3F3}                                                  & \multicolumn{1}{c|}{\cellcolor[HTML]{FFFFFF}Conjunto de Instruções} & CISC            \\ \cline{2-3}
		\rowcolor[HTML]{F3F3F3}
		\multirow{-3}{*}{\cellcolor[HTML]{F3F3F3}Arquitetura}                     & \multicolumn{1}{c|}{\cellcolor[HTML]{F3F3F3}Litografia}             & TSMC 7nm FinFET \\ \hline
		\rowcolor[HTML]{FFFFFF}
		\cellcolor[HTML]{FFFFFF}                                                  & \multicolumn{1}{c|}{\cellcolor[HTML]{FFFFFF}L2}                     & 6MB             \\ \cline{2-3}
		\rowcolor[HTML]{F3F3F3}
		\multirow{-2}{*}{\cellcolor[HTML]{FFFFFF}Cachê}                           & \multicolumn{1}{c|}{\cellcolor[HTML]{F3F3F3}L3}                     & 64MB            \\ \hline
		\rowcolor[HTML]{FFFFFF}
		\multicolumn{2}{|c|}{\cellcolor[HTML]{FFFFFF}TDP}                         & 105W                                                                                  \\ \hline
		\rowcolor[HTML]{F3F3F3}
		\multicolumn{2}{|c|}{\cellcolor[HTML]{F3F3F3}Família de Processadores}    & AMD Ryzen Processors                                                                  \\ \hline
		\rowcolor[HTML]{FFFFFF}
		\multicolumn{2}{|c|}{\cellcolor[HTML]{FFFFFF}Temperatura Maxima}          & 90°C                                                                                  \\ \hline
		\rowcolor[HTML]{F3F3F3}
		\multicolumn{2}{|c|}{\cellcolor[HTML]{F3F3F3}Tamanho do CI (die)}         & 80,7mm²                                                                               \\ \hline
		\rowcolor[HTML]{FFFFFF}
		\multicolumn{2}{|c|}{\cellcolor[HTML]{FFFFFF}Data de Lançamento}          & 05/11/2020                                                                            \\ \hline
	\end{tabular}
	\caption{Tabela de Especificações AMD Ryzen 9 5900X}
	\cite{AdrenaAMD}\cite{siteAMD}\cite{TechAMD}\cite{tomsAMD}\cite{RISCCISC}
	\label{table:amd}
\end{table}




% ----------------------------------------------------------
\section{Intel i9-11900K}
% ----------------------------------------------------------

\begin{table}[!h]
	\centering
	\begin{tabular}{|c|c|c|}
		\hline
		\rowcolor[HTML]{BDBDBD}
		\multicolumn{2}{|c|}{\cellcolor[HTML]{BDBDBD}Especificação}               & Descrição                                                               \\ \hline
		\rowcolor[HTML]{FFFFFF}
		\cellcolor[HTML]{FFFFFF}                                                  & Físicos                     & 8                                         \\ \cline{2-3}
		\rowcolor[HTML]{F3F3F3}
		\multirow{-2}{*}{\cellcolor[HTML]{FFFFFF}Número de Núcleos}               & Threads                     & 16                                        \\ \hline
		\rowcolor[HTML]{FFFFFF}
		\multicolumn{2}{|c|}{\cellcolor[HTML]{FFFFFF}Tamanho da Palavra de Dados} & 64-bits                                                                 \\ \hline
		\rowcolor[HTML]{F3F3F3}
		\multicolumn{2}{|c|}{\cellcolor[HTML]{F3F3F3}GPU Integrada}               & Intel UHD Graphics 750                                                  \\ \hline
		\rowcolor[HTML]{FFFFFF}
		\cellcolor[HTML]{FFFFFF}                                                  & Max Boost                   & 5,3GHz                                    \\ \cline{2-3}
		\rowcolor[HTML]{F3F3F3}
		\multirow{-2}{*}{\cellcolor[HTML]{FFFFFF}Clock}                           & Base                        & 3,5GHz                                    \\ \hline
		\rowcolor[HTML]{FFFFFF}
		\multicolumn{2}{|c|}{\cellcolor[HTML]{FFFFFF}Número de Transistores}      & Não informado pela empresa                                              \\ \hline
		\rowcolor[HTML]{F3F3F3}
		\cellcolor[HTML]{F3F3F3}                                                  & Nome                        & Rocket Lake                               \\ \cline{2-3}
		\rowcolor[HTML]{FFFFFF}
		\cellcolor[HTML]{F3F3F3}                                                  & Conjunto de Instruções      & CISC                                      \\ \cline{2-3}
		\rowcolor[HTML]{F3F3F3}
		\multirow{-3}{*}{\cellcolor[HTML]{F3F3F3}Arquitetura}                     & Litografia                  & 14nm                                      \\ \hline
		\rowcolor[HTML]{FFFFFF}
		\cellcolor[HTML]{FFFFFF}                                                  & Tamanho                     & 16MB                                      \\ \cline{2-3}
		\rowcolor[HTML]{F3F3F3}
		\multirow{-2}{*}{\cellcolor[HTML]{FFFFFF}Cachê}                           & Tecnologia de Gerenciamento & Intel Smart Cache (Ver \ref{cap:smcache}) \\ \hline
		\rowcolor[HTML]{FFFFFF}
		\multicolumn{2}{|c|}{\cellcolor[HTML]{FFFFFF}TDP}                         & 95W                                                                     \\ \hline
		\rowcolor[HTML]{F3F3F3}
		\multicolumn{2}{|c|}{\cellcolor[HTML]{F3F3F3}Família de Processadores}    & 11ª Gração i9                                                           \\ \hline
		\rowcolor[HTML]{FFFFFF}
		\multicolumn{2}{|c|}{\cellcolor[HTML]{FFFFFF}Temperatura Maxima}          & 100°C                                                                   \\ \hline
		\rowcolor[HTML]{F3F3F3}
		\multicolumn{2}{|c|}{\cellcolor[HTML]{F3F3F3}Tamanho do CI (die)}         & 37,5mm²                                                                 \\ \hline
		\rowcolor[HTML]{FFFFFF}
		\multicolumn{2}{|c|}{\cellcolor[HTML]{FFFFFF}Data de Lançamento}          & 1º Quadrimestre de 2021                                                 \\ \hline
	\end{tabular}
	\caption{Tabela de Especificações intel core i9-11900K \cite{intel}}
	\label{table:intel}
\end{table}

\subsection{Smart Cache}\label{cap:smcache}

Smart Cache é um método de cache de nível 2 ou nível 3 para núcleos de execução múltipla, desenvolvido pela Intel.O Smart Cache compartilha a memória cache real entre os núcleos de um processador multi-core. Em comparação com um cache per-core dedicado, a taxa geral de falta de cache diminui quando nem todos os núcleos precisam de partes iguais do espaço do cache. Consequentemente, um único núcleo pode usar o cache de nível 2 ou nível 3, se os outros núcleos estiverem inativos\cite{wiki}