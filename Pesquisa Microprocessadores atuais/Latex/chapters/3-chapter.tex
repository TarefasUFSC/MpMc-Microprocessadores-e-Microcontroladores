\chapter{Conclusão}\label{cap:concs}

Como é observado a partir da comparação entre as tabelas apresentadas, é nítida uma distinção entre os objetivos das fabricantes ao desenvolver os processadores aqui abordados. A Intel, por ter um publico extremamente amplo, focou em um processador top de linha mais genérico tentando cobrir a maioria das pontas soltas, como eficiência energética e poder computacional, contudo, ao tentar ser generalista não consegue competir em certas áreas.


Já a AMD consegue cobrir o nicho de mercado deixado pela Intel, como dito acima, pois no AMD Ryzen 9 5900K é claro, inclusive na propaganda, que o processador tem o foco no público gamer, que não dá tanta relevância a eficiência energética, mas sim para a velocidade que o processador consegue resolver ser problemas nos games. Por isso as especificações da AMD no modelo abordado tendem a ser mais fortes que a Intel neste quesito.